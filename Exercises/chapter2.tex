\part{Foundations}

\chapter{Probabilistic Inference}

\begin{exercise}
\item Bayes rule for medical diagnosis
	\par\smallskip
	After your yearly checkup, the doctor has bad news and good news. The bad news is that you tested
	positive for a serious disease, and the test is \(99\%\) accurate (i.e., the probability of testing positive
	given that you have the disease is \(0.99\), as is the probability of testing negative given that you dont
	have the disease). The good news is that this rare disease, striking only one in \(10,000\) people.
	What are the chances that you actually have the disease?
	\par\smallskip
	Let \(p(Y = 0\mid H = 0)\) be the true negative rate and \(p(Y = 1\mid H = 1)\) be the true positive rate.
	Then the probability matrix will be given by \cref{ch1_ex1}.
	\begin{table}
	\centering
		\begin{tabular}{l | c c}
			& Y = 0 & Y = 1\\ \hline
			H = 0 & 0.99 & 0.01\\
			H = 1 & 0.01 & 0.99
		\end{tabular}
		\caption{The probability matrix for the given rare disease.}
		\label{ch1_ex1}
	\end{table}
	The probability of the prevalence of the disease is \(p(H = 1) = 1 / 10000\) for the population. The 
	probability that we are infected is
	\begin{align}
		p(H = 1\mid Y = 1) 
		&= \frac{p(Y = 1\mid H = 1) p(H = 1)}{p(Y = 1\mid H = 1) p(H = 1) + p(Y = 1\mid H = 0) p(H = 0)}
		\notag\\
		&= \frac{0.99\cdot 0.0001}{0.99\cdot 0.0001 + 0.01 \cdot (1 - 0.0001)}\notag\\
		&= 0.0098\text{ or } 0.98\%\notag
	\end{align}
\item Legal reasoning
	\par\smallskip
	Suppose a crime has been committed. Blood is found at the scene for which there is no innocent explanation.
	It is of a type which is present in \(1\%\) of the population.
	\begin{exercise}
		\item
			The prosecutor claims: "There is a \(1\%\) chance that the defendant would have the crime blood type if he
			were innocent. Thus there is a \(99\%\) chance he is guilty." This is known as the prosecutor's fallacy.
			What is wrong with this argument?
		\item
			The defender claims: "The crime occurred in a city of \(800,000\) people. The blood type would found in
			approximately \(8,000\) people. The evidence has provided a probability of just \(1\) in \(8,000\) that the
			defendant is guilty, and thus has no relevance." This is known as the defender's fallacy. 
			What is wrong with this argument?
	\end{exercise}
\end{exercise}